%% Generated by Sphinx.
\def\sphinxdocclass{report}
\documentclass[letterpaper,10pt,english]{sphinxmanual}
\ifdefined\pdfpxdimen
   \let\sphinxpxdimen\pdfpxdimen\else\newdimen\sphinxpxdimen
\fi \sphinxpxdimen=.75bp\relax
%% turn off hyperref patch of \index as sphinx.xdy xindy module takes care of
%% suitable \hyperpage mark-up, working around hyperref-xindy incompatibility
\PassOptionsToPackage{hyperindex=false}{hyperref}

\PassOptionsToPackage{warn}{textcomp}

\catcode`^^^^00a0\active\protected\def^^^^00a0{\leavevmode\nobreak\ }
\usepackage{cmap}
\usepackage{fontspec}
\usepackage{amsmath,amssymb,amstext}
\usepackage{polyglossia}
\setmainlanguage{english}



\renewcommand{\baselinestretch}{1.4}
\usepackage[cjk]{kotex}
\setmainfont{NanumBarunGothic}[
	BoldFont       = *Bold,
]
\setsansfont{NanumBarunGothic}[
	BoldFont       = *Bold,
]
\setmonofont{NanumGothicCoding}[
	BoldFont       = *-Bold,
]


\usepackage[Sonny]{fncychap}
\ChNameVar{\Large\normalfont\sffamily}
\ChTitleVar{\Large\normalfont\sffamily}
\usepackage{sphinx}

\fvset{fontsize=\small}
\usepackage{geometry}

% Include hyperref last.
\usepackage{hyperref}
% Fix anchor placement for figures with captions.
\usepackage{hypcap}% it must be loaded after hyperref.
% Set up styles of URL: it should be placed after hyperref.
\urlstyle{same}
\addto\captionsenglish{\renewcommand{\contentsname}{EX-pack for Anomaly Detection}}

\usepackage{sphinxmessages}




\title{metatron-doc-user Documentation}
\date{2020년 01월 29일}
\release{0.4.3}
\author{metatron team}
\newcommand{\sphinxlogo}{\vbox{}}
\renewcommand{\releasename}{출시 버전}
\makeindex
\begin{document}

\pagestyle{empty}
\sphinxmaketitle
\pagestyle{plain}
\sphinxtableofcontents
\pagestyle{normal}
\phantomsection\label{\detokenize{index::doc}}



\part{EX-pack for Anomaly Detection}
\label{\detokenize{index:ex-pack-for-anomaly-detection}}

\chapter{Anomaly 확장팩 소개}
\label{\detokenize{part01/index:anomaly}}\label{\detokenize{part01/index::doc}}
이상 탐지 확장팩 Anomaly는 Machine Learning 예측 모델을 기반으로 데이터 흐름의 비정상적인 상황을 감지하여 사용자가 즉각적으로 확인할 수 있도록 도와주는 도구입니다.


\section{기본 원리}
\label{\detokenize{part01/index:basic-principles}}\label{\detokenize{part01/index:id1}}
아래 그림과 같이 Anomaly는 대상 데이터 소스의 집계값을 실시간으로 예측하고 실제 값을 모니터링합니다.
\begin{quote}

\begin{figure}[H]
\centering

\noindent\sphinxincludegraphics{{features_01}.png}
\end{figure}
\end{quote}

여기서 \sphinxstylestrong{Predict}로 표시된 값은 머신러닝 기반으로 예측한 데이터 집계값이고, \sphinxstylestrong{Actual}로 표시된 값은 실제로 모니터링한 결과 값입니다. 아래 그림과 같이 두 값 간의 격차가 커질수록 \sphinxstylestrong{total abnormal score}가 증가하게 됩니다. 즉, 실제치가 예상치와 다르면 데이터 집계값이 그만큼 정상 범위를 벗어났다고 간주하는 것입니다.
\begin{quote}

\begin{figure}[H]
\centering

\noindent\sphinxincludegraphics{{features_02}.png}
\end{figure}
\end{quote}

이 예시에서는 abnormal score가 10점에 도달하면 \sphinxcode{\sphinxupquote{Moderate}} 알람을 발생시키고, 80점에 도달하면 \sphinxcode{\sphinxupquote{Critical}} 알람을 발생시키도록 설정되어 있습니다.

이렇게 발생하는 알람은 다양한 채널로 사용자에게 통보되어, 사용자는 데이터 이상 상황에 즉각 대처할 수 있습니다.


\section{주요 기능}
\label{\detokenize{part01/index:id2}}
Anomaly의 주요 기능은 다음과 같습니다.

\sphinxstylestrong{Machine Learning}
\begin{quote}

머신러닝에 기반한 예측 모델이 자동으로 추천되어 사용자 편의 증대
\end{quote}

\sphinxstylestrong{Alarm \& Report}
\begin{quote}

비정상적인 상황 발생 시 즉각 알람 발동 및 보고서 생성
\end{quote}

\sphinxstylestrong{Analyze}
\begin{quote}

데이터로 차트 생성하고 분석하는 서비스 메타트론 디스커버리와 연계 가능
\end{quote}

\sphinxstylestrong{Link with Learning System}
\begin{quote}

새로운 분석을 적용할 수 있도록 외부 분석 시스템과의 연계를 지원
\end{quote}


\section{구조}
\label{\detokenize{part01/index:id3}}
Anomaly의 메뉴 구성은 다음과 같습니다.
\begin{quote}

\begin{figure}[H]
\centering

\noindent\sphinxincludegraphics{{structure_01}.png}
\end{figure}
\end{quote}

주요 메뉴 간 이동이 쉽고 세부 항목 간 참조 기능도 잘 구축되어 있어, 알람 룰 설정값과 발생한 알람 내역, 그리고 전반적인 알람 현황 간의 유기적인 파악이 용이합니다.


\chapter{알람 룰 만들기}
\label{\detokenize{part02/index:id1}}\label{\detokenize{part02/index::doc}}
Anomaly는 다음의 절차를 순차적으로 수행하도록 안내하여 사용자가 원하는 알람 룰을 쉽게 생성할 수 있도록 지원해줍니다.
\begin{itemize}
\item {} 
{\hyperref[\detokenize{part02/index:select-datasource}]{\sphinxcrossref{\DUrole{std,std-ref}{데이터 소스 선정}}}}

\item {} 
{\hyperref[\detokenize{part02/index:select-columns}]{\sphinxcrossref{\DUrole{std,std-ref}{모니터링할 지표 선택하기}}}}

\item {} 
{\hyperref[\detokenize{part02/index:configure-training}]{\sphinxcrossref{\DUrole{std,std-ref}{트레이닝 기간 설정하기}}}}

\item {} 
{\hyperref[\detokenize{part02/index:select-model}]{\sphinxcrossref{\DUrole{std,std-ref}{모델 선택하기}}}}

\item {} 
{\hyperref[\detokenize{part02/index:alarm-rule-settings}]{\sphinxcrossref{\DUrole{std,std-ref}{알람 룰 조건 설정하기}}}}

\item {} 
{\hyperref[\detokenize{part02/index:complete-rule}]{\sphinxcrossref{\DUrole{std,std-ref}{알람 룰 완성하기}}}}

\end{itemize}


\section{데이터 소스 선정}
\label{\detokenize{part02/index:select-datasource}}\label{\detokenize{part02/index:id2}}
아래와 같이 알람 룰 만들기 절차를 시작하십시오.
\begin{enumerate}
\sphinxsetlistlabels{\arabic}{enumi}{enumii}{}{.}%
\item {} 
Anomaly 홈 우측 상단에 있는 \sphinxstylestrong{Create Alarm Rule} 버튼을 클릭합니다.
\begin{quote}

\begin{figure}[H]
\centering

\noindent\sphinxincludegraphics{{create_rule_01}.png}
\end{figure}
\end{quote}

\item {} 
모니터링하고자 하는 데이터 소스를 선택합니다.
\begin{quote}

\begin{figure}[H]
\centering

\noindent\sphinxincludegraphics{{choose_data_source_01}.png}
\end{figure}
\end{quote}

\end{enumerate}


\section{모니터링할 지표 선택하기}
\label{\detokenize{part02/index:select-columns}}\label{\detokenize{part02/index:id3}}
데이터 소스를 선택하면 다음 화면으로 넘어가면서 좌측에 \sphinxstylestrong{Data} 패널이 열립니다. 이 패널을 이용하여 아래와 같이 모니터링할 지표를 선택하십시오.
\begin{enumerate}
\sphinxsetlistlabels{\arabic}{enumi}{enumii}{}{.}%
\item {} 
\sphinxstylestrong{Measure} 영역에서 알람을 설정하고자 하는 측정값 컬럼을 선택합니다. 클릭한 측정값 컬럼은 Aggregate 선반에 자동으로 옮겨집니다.
\begin{quote}

\begin{figure}[H]
\centering

\noindent\sphinxincludegraphics{{choose_metrics_select_measure_01}.png}
\end{figure}
\end{quote}

\item {} 
필요할 경우 기존 컬럼에 수식을 적용하여 사용자 컬럼을 새로 만들 수도 있습니다. \sphinxstylestrong{Measure} 영역의 우측 상단에서 \sphinxincludegraphics{{icon_custom_column}.png} 버튼을 클릭하여 대화 상자를 열고 사용자 컬럼을 설정하십시오.
\begin{quote}

\begin{figure}[H]
\centering

\noindent\sphinxincludegraphics{{choose_metrics_select_measure_04}.png}
\end{figure}
\end{quote}

\item {} 
Aggregate 선반에 올려진 각 컬럼의 aggregate 타입 항목을 클릭하여 원하는 타입을 선택합니다.
\begin{quote}

\begin{figure}[H]
\centering

\noindent\sphinxincludegraphics{{choose_metrics_select_measure_02}.png}
\end{figure}
\end{quote}

\item {} 
필요할 경우 차원값 컬럼을 기준으로 aggregate 데이터를 분할할 수 있습니다. \sphinxstylestrong{Dimension} 영역에서 분할의 기준으로 삼을 측정값 컬럼에 마우스 커서를 오버한 후 \sphinxincludegraphics{{icon_split}.png} 버튼을 클릭하십시오.
\begin{quote}

\begin{figure}[H]
\centering

\noindent\sphinxincludegraphics{{choose_metrics_select_measure_05}.png}
\end{figure}
\end{quote}

\item {} 
필요할 경우 차원값 컬럼을 기준으로 aggregate 데이터를 필터링할 수 있습니다. \sphinxstylestrong{Dimension} 영역에서 필터를 설정할 측정값 컬럼에 마우스 커서를 오버한 후 \sphinxincludegraphics{{icon_ano_filter}.png} 버튼을 클릭하십시오. 그런 다음, 모니터링하고자 하는 특정 범주들을 선택하십시오.
\begin{quote}

\begin{figure}[H]
\centering

\noindent\sphinxincludegraphics{{choose_metrics_add_filter_02}.png}
\end{figure}
\end{quote}

\end{enumerate}


\section{트레이닝 기간 설정하기}
\label{\detokenize{part02/index:configure-training}}\label{\detokenize{part02/index:id4}}
모니터링할 지표 선택을 마쳤으면, \sphinxstylestrong{Training interval} 패널에서 예측 모델 트레이닝에 사용할 데이터 범위를 선택할 수 있습니다.
\begin{enumerate}
\sphinxsetlistlabels{\arabic}{enumi}{enumii}{}{.}%
\item {} 
모델을 트레이닝시키는 데 사용할 데이터 세트의 주기를 \sphinxstylestrong{Granularity} 선택란에서 선택합니다.
\begin{quote}

\begin{figure}[H]
\centering

\noindent\sphinxincludegraphics{{choose_metrics_select_training_interval_01}.png}
\end{figure}
\end{quote}

\item {} 
모델을 트레이닝시키는 데 사용할 데이터 세트의 기간 범위를 설정합니다.
\begin{quote}

\begin{figure}[H]
\centering

\noindent\sphinxincludegraphics{{choose_metrics_select_training_interval_02}.png}
\end{figure}
\end{quote}

\item {} 
모든 설정을 마쳤으면 \sphinxstylestrong{Next}를 클릭합니다.

\end{enumerate}


\section{모델 선택하기}
\label{\detokenize{part02/index:select-model}}\label{\detokenize{part02/index:id5}}
이제 \sphinxstylestrong{Model} 패널로 넘어가서 어떠한 예측 모델을 사용할지 선택합니다. Anomaly는 주어진 트레이닝 데이터 세트를 이용하여 각각의 모델을 트레이닝시킨 후 그 결과를 산출해줍니다. 아래 두 방법 중 하나를 통해 적합한 예측 모델을 선택하십시오.
\begin{itemize}
\item {} 
기본적으로 우측에 표시되는 정확도 점수(100점 만점)가 가장 높은 모델이 \sphinxstylestrong{Recommend} 표시와 함께 자동 선택됩니다.
\begin{quote}

\begin{figure}[H]
\centering

\noindent\sphinxincludegraphics{{choose_model_01}.png}
\end{figure}
\end{quote}

\item {} 
각 모델 항목 위에 마우스 커서를 오버하면 나타나는 상세 정보를 확인하여 가장 적합한 예측 모델을 직접 선택할 수 있습니다.
\begin{quote}

\begin{figure}[H]
\centering

\noindent\sphinxincludegraphics{{choose_model_03}.png}
\end{figure}
\end{quote}

\end{itemize}


\section{알람 룰 조건 설정하기}
\label{\detokenize{part02/index:alarm-rule-settings}}\label{\detokenize{part02/index:id6}}
사용할 예측 모델을 선택하였으면, \sphinxstylestrong{Condition} 패널에서 알람이 발생하는 조건을 설정할 수 있습니다.
\begin{enumerate}
\sphinxsetlistlabels{\arabic}{enumi}{enumii}{}{.}%
\item {} 
\sphinxstylestrong{Subscribers} 항목의 우측에 있는 \sphinxincludegraphics{{icon_set}.png} 버튼을 클릭하여 대화 상자를 연 후, 알람 발생 시 통보를 받는 대상과 방법을 설정합니다.
\begin{quote}

\begin{figure}[H]
\centering

\noindent\sphinxincludegraphics{{set_alarm_rules_04}.png}
\end{figure}
\end{quote}

\item {} 
아래 각 항목의 설명을 참고하여 알람이 발동되는 시기를 설정합니다.
\begin{quote}

\begin{figure}[H]
\centering

\noindent\sphinxincludegraphics{{set_alarm_rules_01}.png}
\end{figure}
\begin{itemize}
\item {} 
\sphinxstylestrong{Alarm Start:} 알람을 개시할 때를 설정합니다. 이 설정값에 해당하는 시간 이후부터 알람이 개시됩니다.

\item {} 
\sphinxstylestrong{Alarm Interval:} 알람의 조건이 충족되었을 때 알람을 발생시키는 주기를 설정합니다.

\end{itemize}
\end{quote}

\item {} 
아래 각 항목의 설명을 참고하여 모니터링 대상 데이터의 abnormal score에 따른 알람 발동 조건을 설정합니다. 기본적으로 하나의 조건이 주어지며, \sphinxstylestrong{+ Add Condition} 버튼을 클릭하면 조건을 추가할 수 있습니다.
\begin{quote}

\begin{figure}[H]
\centering

\noindent\sphinxincludegraphics{{set_alarm_rules_02}.png}
\end{figure}
\begin{itemize}
\item {} 
\sphinxstylestrong{Severity:} 주어진 조건에 해당하는 알람의 심각도를 설정합니다.

\item {} 
\sphinxstylestrong{Threshold:} abnormal score가 이 설정값을 초과하면 데이터 이상 상태로 간주됩니다.

\item {} 
\sphinxstylestrong{Frequency:} abnormal score가 한계값을 초과하는 빈도가 어떠할 때 알람을 발생시킬지 결정합니다. 예를 들어, 《3 within 5 minute》로 설정한 경우에는 abnormal score가 5분 안에 3회 이상 한계값을 초과하면 알람이 발생합니다.

\end{itemize}
\end{quote}

\item {} 
모든 설정을 마쳤으면 \sphinxstylestrong{Next}를 클릭합니다.

\end{enumerate}


\section{알람 룰 완성하기}
\label{\detokenize{part02/index:complete-rule}}\label{\detokenize{part02/index:id7}}
알람 룰 설정이 끝났으면 아래와 같이 알람 룰 만들기 절차를 마무리합니다.
\begin{enumerate}
\sphinxsetlistlabels{\arabic}{enumi}{enumii}{}{.}%
\item {} 
알람 룰의 이름과 설명을 기입한 후 \sphinxstylestrong{Done} 버튼을 클릭합니다.
\begin{quote}

\begin{figure}[H]
\centering

\noindent\sphinxincludegraphics{{complete_alarm_rules_01}.png}
\end{figure}
\end{quote}

\item {} 
생성된 알람 룰은 알람 룰 리스트의 최상단에 노출되고, 첫 알람 수행이 있기 전까지 \sphinxstylestrong{Prepare} 상태로 표시됩니다.
\begin{quote}

\begin{figure}[H]
\centering

\noindent\sphinxincludegraphics{{complete_alarm_rules_02}.png}
\end{figure}
\end{quote}

\end{enumerate}


\chapter{통계}
\label{\detokenize{part03/index:statistics}}\label{\detokenize{part03/index:id1}}\label{\detokenize{part03/index::doc}}
\sphinxstylestrong{Statistics} 탭 메뉴에서는 발생한 알람의 전반적인 통계를 보여줍니다. 이 페이지에서는 사용자가 지금까지 발생한 알람의 현황을 다각도로 파악할 수 있도록 중요도, 알람 발생 시기, 알람 룰 등의 다양한 기준으로 통계를 산출하여 제시합니다.

페이지 기본 구성은 다음과 같습니다.
\begin{quote}

\begin{figure}[H]
\centering

\noindent\sphinxincludegraphics{{overview_01}.png}
\end{figure}
\begin{itemize}
\item {} 
\sphinxstylestrong{Alarm Distribution by Severity:} 심각도별 알람 발생 비중을 보여줍니다.

\item {} 
\sphinxstylestrong{Alarm Count per Time:} 시간대별 알람 빈도를 보여줍니다.

\item {} 
\sphinxstylestrong{Top 5 Subscribers:} 가장 많은 알람을 통보받은 사용자 5명을 보여줍니다.

\item {} 
\sphinxstylestrong{Top 5 Alarm rules:} 가장 많은 알람을 일으킨 알람 룰 5개를 보여줍니다.

\item {} 
\sphinxstylestrong{Latest Alarms:} 가장 최근에 발생한 알람들을 보여줍니다.

\end{itemize}
\end{quote}

페이지 상단의 기간 설정 메뉴를 이용하면 통계를 산출하는 기준 기간을 변경할 수 있습니다.
\begin{quote}

\begin{figure}[H]
\centering

\noindent\sphinxincludegraphics{{overview_02}.png}
\end{figure}
\end{quote}


\chapter{얄람 룰 내역 열람·수정하기}
\label{\detokenize{part04/index:id1}}\label{\detokenize{part04/index::doc}}
\sphinxstylestrong{Alarm Rule} 탭 메뉴에서는 등록된 알람 룰을 열람·수정할 수 있습니다. 또한 이 메뉴에서는 선택한 예측 모델에 따라 산출되고 있는 데이터 abnormal score 현황도 쉽게 파악할 수 있습니다.

알람 룰 메뉴는 다음의 두 가지 페이지로 구성되어 있습니다.
\begin{itemize}
\item {} 
{\hyperref[\detokenize{part04/index:alarm-rule-list}]{\sphinxcrossref{\DUrole{std,std-ref}{알람 룰 리스트}}}}

\item {} 
{\hyperref[\detokenize{part04/index:alarm-rule-details}]{\sphinxcrossref{\DUrole{std,std-ref}{알람 룰 상세}}}}

\end{itemize}


\section{알람 룰 리스트}
\label{\detokenize{part04/index:alarm-rule-list}}\label{\detokenize{part04/index:id2}}
\sphinxstylestrong{Alarm Rule} 탭으로 들어가면 현재 등록된 알람 룰들을 열거하여 보여줍니다.
\begin{quote}

\begin{figure}[H]
\centering

\noindent\sphinxincludegraphics{{complete_alarm_rules_021}.png}
\end{figure}
\end{quote}

리스트에 표시되는 정보는 아래와 같으며, 이를 기준으로 열거할 룰을 필터링하거나 검색할 수 있습니다.
\begin{itemize}
\item {} 
\sphinxstylestrong{Current Status:} 해당 룰에 따른 모니터링 결과 상태

\item {} 
\sphinxstylestrong{Alarm Rule Name:} 해당 룰의 이름

\item {} 
\sphinxstylestrong{DataSource:} 모니터링 대상 데이터 소스

\item {} 
\sphinxstylestrong{Measure:} 모니터링 대상 측정값 컬럼

\item {} 
\sphinxstylestrong{Alarm Interval:} 알람 발생 주기

\item {} 
\sphinxstylestrong{Condition:} 해당 룰에 적용된 알람 발생 조건의 개수

\item {} 
\sphinxstylestrong{Alarm:} 해당 룰에 의해 발생한 알람의 수

\item {} 
\sphinxstylestrong{Running:} 해당 룰의 모니터링 활성 여부

\item {} 
\sphinxstylestrong{Updated:} 해당 룰을 마지막으로 업데이트한 시간과 사용자

\end{itemize}


\section{알람 룰 상세}
\label{\detokenize{part04/index:alarm-rule-details}}\label{\detokenize{part04/index:id3}}
알람 룰 리스트에 열거된 항목 중 하나를 선택하면 해당 알람 룰에 대한 상세 정보를 열람하고 설정을 수정할 수 있습니다. 화면 좌측에서는 모니터링 현황을 시각화하여 보여주고, 우측에는 알람 룰 조건 설정값이 표시됩니다.
\begin{quote}

\begin{figure}[H]
\centering

\noindent\sphinxincludegraphics{{alarm_rule_detail_01}.png}
\end{figure}
\end{quote}

모니터링 현황 영역 상단에는 화면에 보여주는 모니터링 기간 설정값이 표시되어 있습니다. \sphinxincludegraphics{{icon_period_edit}.png} 아이콘을 클릭하면 기간 설정값을 변경할 수 있습니다.
\begin{quote}

\begin{figure}[H]
\centering

\noindent\sphinxincludegraphics{{alarm_rule_detail_02}.png}
\end{figure}
\end{quote}

알람 룰 조건 설정 영역에서는 기존에 설정된 알람 룰 설정값을 수정할 수 있습니다. 자세한 내용은 {\hyperref[\detokenize{part02/index:alarm-rule-settings}]{\sphinxcrossref{\DUrole{std,std-ref}{알람 룰 조건 설정하기}}}} 항목을 참조하십시오.
\begin{quote}

\begin{figure}[H]
\centering

\noindent\sphinxincludegraphics{{alarm_rule_detail_03}.png}
\end{figure}
\end{quote}

우측 끝단에서 \sphinxincludegraphics{{icon_alarm_history}.png} 버튼을 누르면 \sphinxstylestrong{Conditions} 패널이 \sphinxstylestrong{Alarm History} 패널로 전환되어 지금까지 발생한 알람 이력을 보여줍니다(다시 \sphinxincludegraphics{{icon_rule_settings_edit}.png} 버튼을 누르면 \sphinxstylestrong{Conditions} 패널로 되돌아옵니다).
\begin{quote}

\begin{figure}[H]
\centering

\noindent\sphinxincludegraphics{{alarm_rule_detail_04}.png}
\end{figure}
\end{quote}


\chapter{알람 내역 열람하기}
\label{\detokenize{part05/index:id1}}\label{\detokenize{part05/index::doc}}
\sphinxstylestrong{Alarm} 탭 메뉴에서는 지금까지 발생한 알람 내역을 확인할 수 있습니다. 알람의 전체적인 현황을 보여주는 {\hyperref[\detokenize{part03/index:statistics}]{\sphinxcrossref{\DUrole{std,std-ref}{통계}}}} 페이지와는 다르게 이 메뉴에서는 보다 개별적인 알람들을 열람하고 탐색하는 데 최적화된 UI를 제공합니다.

이 메뉴는 다음의 두 가지 페이지로 구성되어 있습니다.
\begin{itemize}
\item {} 
{\hyperref[\detokenize{part05/index:alarm-list}]{\sphinxcrossref{\DUrole{std,std-ref}{알람 리스트}}}}

\item {} 
{\hyperref[\detokenize{part05/index:alarm-details}]{\sphinxcrossref{\DUrole{std,std-ref}{알람 상세}}}}

\end{itemize}


\section{알람 리스트}
\label{\detokenize{part05/index:alarm-list}}\label{\detokenize{part05/index:id2}}
\sphinxstylestrong{Alarm} 탭으로 들어가면 현재까지 발생한 알람들을 열거하여 보여줍니다. 화면 상단에 있는 \sphinxstylestrong{Alarm rule} / \sphinxstylestrong{Timeline} 선택 박스를 이용하여, 알람 리스트를 알람 룰 기준으로 정렬할 수도 있고, 발생한 시간 기준으로 정렬할 수도 있습니다.
\begin{itemize}
\item {} 
\sphinxstylestrong{Alarm rule} (알람 룰 기준으로 정렬)
\begin{quote}

\begin{figure}[H]
\centering

\noindent\sphinxincludegraphics{{alarm_list_01}.png}
\end{figure}
\end{quote}

\item {} 
\sphinxstylestrong{Timeline} (발생 시간 기준으로 정렬)
\begin{quote}

\begin{figure}[H]
\centering

\noindent\sphinxincludegraphics{{alarm_list_03}.png}
\end{figure}
\end{quote}

\end{itemize}

카테고리 맨 끝에 있는 \sphinxstylestrong{+ Load more}를 클릭하면 해당 카테고리 내 더 많은 알람 항목을 보여줍니다.
\begin{quote}

\begin{figure}[H]
\centering

\noindent\sphinxincludegraphics{{alarm_list_02}.png}
\end{figure}
\end{quote}


\section{알람 상세}
\label{\detokenize{part05/index:alarm-details}}\label{\detokenize{part05/index:id3}}
알람 리스트에 열거된 항목 중 하나를 선택하면 해당 알람에 대한 상세 정보를 열람할 수 있습니다. 아래는 알람 상세 페이지의 각 영역별 설명입니다.


\subsection{Alarm Info 영역}
\label{\detokenize{part05/index:alarm-info}}
이 영역에서는 해당 알람의 심각도와 발생 시각, 그리고 이 알람을 발생시킨 룰의 설정값을 보여줍니다.
\begin{quote}

\begin{figure}[H]
\centering

\noindent\sphinxincludegraphics{{alarm_detail_01}.png}
\end{figure}
\end{quote}


\subsection{알람 현황 일람 상자}
\label{\detokenize{part05/index:id4}}
이 영역에서는 해당 알람의 발생 현황을 보여줍니다. 정해진 주기에 따라 알람이 연속적으로 발생하면 1개의 알람 항목으로 계속 유지됩니다. 아래 그림 예시에서는 알람이 4번의 주기 동안 연속적으로 발생하였고(\sphinxstylestrong{Alarms}), 주기가 1분이었기 때문에 4건의 알람이 총 4분 동안 지속된 것입니다(\sphinxstylestrong{Elapsed Time}).
\begin{quote}

\begin{figure}[H]
\centering

\noindent\sphinxincludegraphics{{alarm_detail_02}.png}
\end{figure}
\end{quote}


\subsection{Alarm History 영역}
\label{\detokenize{part05/index:alarm-history}}
이 영역에서는 해당 알람에 적용된 알람 룰에 의해 발생한 알람의 이력을 보여줍니다.
\begin{quote}

\begin{figure}[H]
\centering

\noindent\sphinxincludegraphics{{alarm_detail_03}.png}
\end{figure}
\end{quote}


\subsection{Chart View 탭}
\label{\detokenize{part05/index:chart-view}}
이 탭 영역에서는 해당 알람 구간 내에서 모니터링한 aggregate 데이터의 abnormal score 추이를 차트로 보여줍니다. 여기서는 각 조건별 점수 한계값에 도달하여 상응하는 알람(\sphinxcode{\sphinxupquote{Critical}}, \sphinxcode{\sphinxupquote{Major}}, \sphinxcode{\sphinxupquote{Moderate}}, \sphinxcode{\sphinxupquote{Low}})을 일으킨 발생 건도 보고해줍니다. 차트 산출 방식에 관해서는 {\hyperref[\detokenize{part01/index:basic-principles}]{\sphinxcrossref{\DUrole{std,std-ref}{기본 원리}}}} 항목을 참조하십시오.
\begin{quote}

\begin{figure}[H]
\centering

\noindent\sphinxincludegraphics{{alarm_detail_04}.png}
\end{figure}
\begin{itemize}
\item {} 
\sphinxstylestrong{Total abnormal score:} 알람 룰에 포함된 모든 측정값 컬럼에 대한 abnormal score를 보여줍니다.

\item {} 
\sphinxstylestrong{Chart by measures:} 알람 룰에 포함된 각 개별 측정값 컬럼 데이터의 예측치와 실제치의 추이를 보여줍니다.

\end{itemize}
\end{quote}


\subsection{Table View 탭}
\label{\detokenize{part05/index:table-view}}
이 탭 영역에서는 각 알람 발생 건별로 데이터 실제치와 예측치, 그리고 abnormal score를 보여줍니다.
\begin{quote}

\begin{figure}[H]
\centering

\noindent\sphinxincludegraphics{{alarm_detail_05}.png}
\end{figure}
\end{quote}



\renewcommand{\indexname}{색인}
\printindex
\end{document}